\documentclass[12pt,]{article}
\usepackage{lmodern}
\usepackage{amssymb,amsmath}
\usepackage{ifxetex,ifluatex}
\usepackage{fixltx2e} % provides \textsubscript
\ifnum 0\ifxetex 1\fi\ifluatex 1\fi=0 % if pdftex
  \usepackage[T1]{fontenc}
  \usepackage[utf8]{inputenc}
\else % if luatex or xelatex
  \ifxetex
    \usepackage{mathspec}
    \usepackage{xltxtra,xunicode}
  \else
    \usepackage{fontspec}
  \fi
  \defaultfontfeatures{Mapping=tex-text,Scale=MatchLowercase}
  \newcommand{\euro}{€}
    \setmainfont{Calibri Light}
\fi
% use upquote if available, for straight quotes in verbatim environments
\IfFileExists{upquote.sty}{\usepackage{upquote}}{}
% use microtype if available
\IfFileExists{microtype.sty}{%
\usepackage{microtype}
\UseMicrotypeSet[protrusion]{basicmath} % disable protrusion for tt fonts
}{}
\usepackage[margin=1in]{geometry}
\ifxetex
  \usepackage[setpagesize=false, % page size defined by xetex
              unicode=false, % unicode breaks when used with xetex
              xetex]{hyperref}
\else
  \usepackage[unicode=true]{hyperref}
\fi
\hypersetup{breaklinks=true,
            bookmarks=true,
            pdfauthor={},
            pdftitle={Can we detect decline in diversity using molecular phylogenies?},
            colorlinks=true,
            citecolor=blue,
            urlcolor=blue,
            linkcolor=magenta,
            pdfborder={0 0 0}}
\urlstyle{same}  % don't use monospace font for urls
\setlength{\parindent}{0pt}
\setlength{\parskip}{6pt plus 2pt minus 1pt}
\setlength{\emergencystretch}{3em}  % prevent overfull lines
\providecommand{\tightlist}{%
  \setlength{\itemsep}{0pt}\setlength{\parskip}{0pt}}
\setcounter{secnumdepth}{0}

%%% Use protect on footnotes to avoid problems with footnotes in titles
\let\rmarkdownfootnote\footnote%
\def\footnote{\protect\rmarkdownfootnote}

%%% Change title format to be more compact
\usepackage{titling}

% Create subtitle command for use in maketitle
\newcommand{\subtitle}[1]{
  \posttitle{
    \begin{center}\large#1\end{center}
    }
}

\setlength{\droptitle}{-2em}
  \title{Can we detect decline in diversity using molecular phylogenies?}
  \pretitle{\vspace{\droptitle}\centering\huge}
  \posttitle{\par}
  \author{}
  \preauthor{}\postauthor{}
  \predate{\centering\large\emph}
  \postdate{\par}
  \date{July 15, 2015}


% Redefines (sub)paragraphs to behave more like sections
\ifx\paragraph\undefined\else
\let\oldparagraph\paragraph
\renewcommand{\paragraph}[1]{\oldparagraph{#1}\mbox{}}
\fi
\ifx\subparagraph\undefined\else
\let\oldsubparagraph\subparagraph
\renewcommand{\subparagraph}[1]{\oldsubparagraph{#1}\mbox{}}
\fi

\begin{document}
\maketitle

\section{Introduction}\label{introduction}

Over the past two decades, it is possible to see the rise and
establishment of molecular phylogenies as important information used to
estimate rates of diversification (Nee et al. (1994), Ape, Geiger,
Diversitree, BAMM). The use of those phylogenies have allowed
researchers to evaluate the diversification dynamics of several types of
organisms, and especially have allowed the study of fossil-poor clades
such as birs for instance. However, some skepticism regarding the extent
to which we can rely only on the molecular phylogenies was raised, due
to the very small values of extinction rates commonly obtained when
using only this type of data (Rabosky (2010), Quental and Marshall
(2011)).

\begin{itemize}
\item
  Over the last decade researchers are using molecular phylogenies of
  extant lineages to estimate rates of diversification (speciation and
  extinction). Skepticism comes from Rabosky 2010, Quental \& Marshall
  2011, due to extinction rates being frequently estimated with very
  small values using molecular phylogenies.
\item
  However, according to the fossil record most lineages are already
  extinct and many of them are in a declining phase (examples: canids,
  rhinos, hyena, horses).
\item
  Lineages are characterized by an expansion and a decline phase and it
  is still an open question whether we can detect it or not using
  molecular phylogenies (Quental \& Marshall 2011 suggest with a simple
  approach -- gamma statistics -- that it is difficult to distinguish
  between decline and stable diversity; Morlon \emph{et al.} 2011, with
  more sophisticated methods were able to detect decline; but discuss
  the limitation of their results).
\item
  However, how general are the results from Morlon \emph{et al.}? Lack
  of papers that have used these methods, and even greater lack of
  papers that actually found decline with these methods.
\item
  Curiously BiSSE-like models suggest that some traits might be
  associated with negative diversification rates (Goldberg \emph{et al.}
  2010, Burin \emph{et al. 2015}, among others).
\item
  Additionally, more recently Beaulieau \& O'Meara shown by revisiting
  Rabosky's 2010 that extinction rates can be reasonably estimated in
  some scenarios, although these scenarios are limited by not including
  decline in diversification.
\item
  New methods (Morlon \emph{et al.} (2011) and Rabosky (2014)) allow for
  extinction to be higher than speciation although none of these two
  methods were tested thoroughly in their capacity of detecting declines
  in different points in time. Liow et al. 2010 show that the signature
  of a given diversity dynamics might change as time goes by.
\item
  Here we present a broad analysis to compare their ability to detect
  diversity decline in different evolutionary scenarios over different
  viable parameter combinations within a comprehensive parameter space.
  Mention that our aim is not only to assess the potential of both
  methods in estimating the true rates, but also how do both perform
  when evaluating the diversity trajectory over time.
\item
  In summary, most of the groups analyzed would still be expanding which
  is in disagreement with the trend suggested by the fossil record.
\end{itemize}

\section{Goals}\label{goals}

Our goal was to explore the performance of two recently proposed
methods, Morlon \emph{et al.} (2011) and Rabosky (2014) (BAMM) in
detecting diversity declines using molecular phylogenies.

\section{Material and Methods}\label{material-and-methods}

\subsection{Parameter space
exploration}\label{parameter-space-exploration}

We divided our simulations into two scenarios: the first scenario had
exponential decline on speciation rates and constant extinction rates
through time (BVARDCST), whereas the second scenario had constant
speciation rates and exponential increase on extinction rates over time
(BCSTDVAR). For these two scenarios, four parameters were combined
according to each scenario: for BVARDCST, we used two parameters for
speciation rates (initial speciation \(\lambda_0\) and decaying rate
\(\alpha\)) and one parameter for extinction rates -- \(\mu\). In the
second scenario (BCSTDVAR), we used one parameter for speciation rates
(\(\lambda\)) and two for extinction rates (initial extinction \(\mu_0\)
and decaying rate \(\beta\)). These values were sampled randomly for
each of the simulations (simulation process described below) from
uniform distributions bounded by the values presented in table 1.

\subsection{Diversification scenarios and
simulations}\label{diversification-scenarios-and-simulations}

The simulation process of the 2000 simulated trees for each scenario
consisted of 6 steps, as follows. (1) Parameter values (\(\lambda\),
\(\alpha\), \(\mu\) and \(\beta\)) were randomly samples from a uniform
distribution with the limits indicated in table 1. (2) The expected time
of the initiation of the decline phase was calculated (\(t_{max}\) when
speciation = extinction). We also calculated the expected peak species
diversity at \(t_{max}\). (3) We then estimated the time necessary for
losing 80\% of the peak diversity. The final species diversity values
were forced to lie between 10 and 500 species at the end of the
simulation, due to limitations imposed by working with trees too small
(low statistical power) or too big (high computational memory/time
demand). If the expected final diversity was outside of these limits,
the sampled parameter values were stored and discarded, and steps 1-3
were repeated. The time needed for the species diversity to drop to 80\%
of peak diversity was obtained by numerical approximation, since it is
not possible to analytically integrate the speciation and extinction
functions. (4) The time calculated in the previous step was then used in
the simulation function. The function is available at
\url{http://github.com/gburin/labmeme/bamm_rpanda}, and its initial
version was kindly provided by Dr.~Helene Morlon. The function simulates
trees according to time-varying speciation and/or extinction rates, and
stores the full resulting phylogeny that contains both extant and
extinct species. We modified it to limit the maximum total diversity to
20000 species; if at some point the simulation reached this limit the
simulation was interrupted, the parameters were stored and discarded and
steps 1-4 were repeated; additionally, some simulated trees went fully
extinct before the set time: in this case, the parameter values were
also stored and discarded, and the steps 1-4 were repeated. All
parameter combinations (valid, extinct or ``exploded'' were stored in
separate files, and used to explore the properties of the parameter
space.

After simulating all trees for each scenario, three final steps were
performed. (5) We used the estimated diversity estimates for the whole
life of each simulated tree to calculate, analogous to step 3, the times
needed for the loss of 50\% and 20\% of peak species diversity.
Furthermore, to check for false positives, we also calculated the time
needed to reach 20\% less of peak diversity but still on the increase
phase, were the methods are expected to not detect decline in diversity.
(6) All trees had then their extinct species pruned to give us the
corresponding molecular phylogenies.

\subsection{Model fitting and parameter
estimation}\label{model-fitting-and-parameter-estimation}

We used two recently described methods (RPANDA and BAMM) to retrieve
information about speciation and extinction rates for the resulting
molecular trees. RPANDA estimates the parameters of the functions that
describe the variation of speciation and extinction rates through time,
whereas BAMM provides average instantaneous rates for each time step
within the duration of the tree. Both models were thus fitted to each
one of the trees in both scenarios. RPANDA models were fitted within the
R environment, whereas BAMM is coded in python. The two models provide
distinct types of information: while RPANDA returns parameter estimates
for the model(s) one is evaluating, BAMM generates point mean rate
estimates for discrete time intervals along a tree history. Thus, in
order to compare both models, we analyzes only the initial and final
values of the varying rate and of net diversification as well as the
constant rate values.

\section*{References}\label{references}
\addcontentsline{toc}{section}{References}

\hyperdef{}{refs}{\label{refs}}
\hyperdef{}{ref-Nee1994}{\label{ref-Nee1994}}
Nee S., May R.M., Harvey P.H. 1994. The reconstructed evolutionary
process. Philosophical transactions of the Royal Society of London.
Series B, Biological sciences.

\hyperdef{}{ref-Quental2011}{\label{ref-Quental2011}}
Quental T.B., Marshall C.R. 2011. The molecular phylogenetic signature
of clades in decline. PloS one. 6:e25780.

\hyperdef{}{ref-Rabosky2010}{\label{ref-Rabosky2010}}
Rabosky D.L. 2010. Extinction rates should not be estimated from
molecular phylogenies. Evolution. 64:1816--1824.

\end{document}
